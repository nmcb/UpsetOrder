\documentclass{sigplanconf}
\usepackage{stmaryrd}
\usepackage{MnSymbol}

\usepackage{ifpdf}
\usepackage[usenames]{color}
\definecolor{citationcolour}{rgb}{0,0.4,0.2}
\definecolor{linkcolour}{rgb}{0,0,0.8}
\usepackage{hyperref}
\hypersetup{colorlinks=true,
            urlcolor=linkcolour,
            linkcolor=linkcolour,
            citecolor=citationcolour,
            pdftitle=Dependent Types for Your New World Order,
            pdfauthor={Conor McBride},
            pdfkeywords={}}  



\usepackage{pig}


\begin{document}
\title{Upsetting The New World Order}
\authorinfo{Conor McBride}
           {University of Strathclyde}
           {conor@strictlypositive.org}
\maketitle

\begin{abstract}
  In a world where modelling information flow (to allow or prevent it)
  is becoming increasingly critical, type systems which account only
  for \emph{what} stuff consists of, not \emph{where} it is,
  \emph{who} knows it, \emph{when} it is available, or other such
  considerations of time and space, are sadly separated from key
  concerns.  Dependent type systems generally require a Kripke
  semantics to account for their contextualisation, so it might as
  well be an interesting Kripke semantics. There should probably be a
  sentence between the first two sentences and this sentence should
  probably not exist. There should probably be a sentence after this
  sentence but it does not exist yet.
\end{abstract}


\section{Introducing Worlds}

Different things exist in different \emph{worlds}. Your computer has
different data from my computer. The data available to the typechecker
of my program are different from the data available to my program at
runtime. Because I am a dependently typed programmer, abstract
approximants to all data flow from runtime code to the typechecker,
but, like Nathan Mishra-Linger and Tim
Sheard~\cite{DBLP:conf/fossacs/Mishra-LingerS08}, I might like to
ensure that information flow from typechecker to runtime is
restricted in order to obtain healthy \emph{erasure} properties and
acknowledge the fact that quantities which exist only to sustain
specifications need not occupy memory once the hard work starts.

\newcommand{\wto}{\rightarrowtail}
\newcommand{\hb}{\!:\!}
\newcommand{\Ty}{\pentagram}
\newcommand{\earth}{\bullet}
\newcommand{\heaven}{\ast}
\newcommand{\tn}[1]{\vdash #1 \;}

Accordingly, let $W$ be a preorder of `worlds' $u,v,w$ with arrows
$\wto$. (Of course, $W$ is thus a category, and I shall prefer to use
categorical terminology.)  These worlds represent different places
where data can exist and computation can happen.  For
example\footnote{I say `example', despite my preference for
  categorical terminology.}, $W=\{\earth,\heaven\}$ with $\earth \wto
\heaven$ is the simplest nontrivial such system, modelling has runtime
data on `earth', $\earth$, and typechecking in `heaven', $\heaven$,
with information flowing only upwards.

The preorder structure tells us between which worlds information can
flow without explicit communication: that is, which worlds are
\emph{in scope} for which other worlds. Worlds will show up in
judgments: everything we construct will be constructed in a particular
world. Worlds will show up in contexts: every variable we assume must
be somewhere, as well as some kind of thing. Context entries look like
$u\,x\hb S$, where $u$ is the world $x$ comes from and $S$ is its
type. The key rule is this:
\[
\Rule{\Gamma,u\,x\hb S,\Delta\vdash\textsc{valid}\quad u\wto w}
  {\Gamma,u\,x\hb S,\Delta\tn{w}\,x\in S}
\]
That is, $x$ can be used for constructions in any world $w$ (now
part of the turnstile) to which $x$ may travel from its `home' world $u$.

Many type systems have some implicit separation of worlds baked into
them, and it is very often the case that the worlds something may
inhabit are closely related with the kind of thing it is. For example,
last century, we became too used to the idea that runtime stuff and
the stuff of types were wholly distinct. It is liberating to discard
this correlation and treat \emph{what} and \emph{where} as separable
concerns. Motivation comes also from last century, when Hoare logic
separated runtime `program variables' from `ghost variables', the
stuff exclusively of \emph{specifications}, e.g. `on exit, program
variable $p$ equals ghost variable $g$': $p$ and $g$ come from different
worlds but stand for the same kinds of stuff.

This paper is concerned not to establish a specific world order with a
specific motivation, but rather to develop the concept in general and
provide a framework for establishing dependent type systems with some
sort of information flow. The point is not that there is one best
world order, but that it is useful to consider which orders may be
locally appropriate and how to transport constructions between them.
Some useful world orders to bear in mind include the following.
\[\begin{array}{c@{\quad}|@{\quad}c@{\quad}|@{\quad}c@{\quad}|@{\quad}c@{\quad}|@{\quad}c@{}c@{}c@{}l}
\mbox{(i)} & \mbox{(ii)} & \mbox{(iii)} & \mbox{(iv)} &
\multicolumn{3}{c}{\mbox{(v)}} \\
            &               &               &     \ast 
\\
            &               &   \ast        & \nearrowtail\quad\nwarrowtail
  & \ast \\
\circledast & \bullet\;\ast & \uparrowtail  & \bullet_0\qquad\;\;\bullet_1 
  & \uparrowtail &\nwarrowtail & & \;\ddots \nwarrowtail \qquad\ddots\\
            &               &   \bullet     & \nwarrowtail\quad\nearrowtail 
  & \;\bullet_0&\rightarrowtail&\bullet_1&\rightarrowtail\;\cdots\;\bullet_n\rightarrowtail\cdots \\
            &               &               &     \blacksquare
\end{array}\]
Respectively, we have: (i) dependent type theory as we know it, all in
one world; (ii) System F-like arrangements with no flow
between terms and types; (iii) our running example, dependent type
theory with erasure; (iv) coordinated computation in two separate
sites with some common knowledge; (v) computation in a succession of
stages, as used by Paula Severi and Fer-Jan de Vries to model
productivity of stream computations~\cite{DBLP:conf/icfp/SeveriV12}.


\section{Upsets of Worlds}

An \emph{upset} of $W$ is a subset $U \subseteq W$ such that whenever
$u\in U$ and $u\wto w$, $w\in U$. Once information has arrived in an
upset $U$, it is trapped in $U$. Correspondingly, we shall obtain an
erasure operation, removing the parts of constructions in any such
$U$, as the rest of those constructions cannot make use of
$U$-information.\footnote{We can draw $U$-shaped curves
around upsets in diagrams like those above.} If, in (ii), we erase
$\{\ast\}$, we translate from Church-style to
Curry-style. In (iii), erasing $\{\ast\}$ gives us a translation to a
system like Miquel's Implicit Calculus of
Constructions~\cite{DBLP:conf/tlca/Miquel01}. In (iv), we might erase
$\{\ast,\bullet_i\}$ to obtain the code to run in site $1-i$.
In (v), erasing $\{\ast\}\cup\{\bullet_i|i\ge n\}$ obtains
just the first $n$ stages.

However, upsets are not just for erasure. Let $\Ty$ be an upset of
$W$, namely the subset of worlds in which the construction of types is
permitted. If we want to construct types at all, we should probably
ensure that $\Ty$ is nonempty. The extent to which a system is
`dependently typed' is in some way characterized by the information
flow to worlds in $\Ty$.  In our example system, take
$\Ty=\{\heaven\}\subseteq\{\earth,\heaven\}=W$. This system has `full
dependent types', because all things are in scope for typechecking,
but supports erasure in that information which exists \emph{only} for
typechecking purposes cannot flow to runtime. Note that we can and
should consider the alternative of taking $\Ty=W$, which would allow
typecase for types in $\bullet$ whilst retaining parametricity with
respect to types in $\ast$.

In order to account for valid type formation, we shall need a monotone
operator $\Ty\cdot : W \to \Ty$ which maps each world $w$ to the world
in which the types of $w$-constructions are constructed. For our
example, clearly $\Ty w = \ast$, but more subtle variants are
conceivable.


\section{Quantifiers are not Worlds: Types are Portable}

\newcommand{\Q}{}
 Let us consider functions and their types.  There is a set $Q$ of
\emph{quantifiers}, characterising the varieties of functional
abstraction which are possible. Each quantifier acts as an endofunctor on
worlds, $\Q{q}w$: If $u \wto w$, then $\Q{q}u \wto \Q{q}w$.  This action explains
how functions change their usage as they move between worlds.

\newcommand{\V}{\mathit}
\newcommand{\F}{\mathsf}
\newcommand{\D}{\mathsf}
Our two-world example provides motivation for this choice. We should
expect that $\F{+}$ exists at runtime and can add two runtime numbers,
but we should be sorely disappointed if $\F{+}$, on arrival at the type
level, could not also be used for numbers
existing only at the type level. We should be able to type vector
concatentation without insisting that the lengths of the vectors
exist at run time. Let us therefore equip our example system with
\[
Q=\{\Q{\Pi},\Q{\cap}\}\qquad
\begin{array}[t]{l|cc}
\Q{q}w & \earth & \heaven \\
\hline
\Pi & \earth & \heaven \\
\cap & \heaven & \heaven
\end{array}
\]
and thus ensure that all functions used in types take type-level
values (or runtime values, which can flow to the type-level).

\newcommand{\qt}[3]{(\Q{#1}\,#2\hb #3)\to}
\newcommand{\pit}{\qt\Pi}
\newcommand{\cat}{\qt\cap}
The formal treatment will be more carefully bidirectional, but
informally, the relevant rules are.
\[\begin{array}{c}
\Rule{\Gamma,\Q{q}w\,\V{x}\hb S \vdash w\, t : T}
  {\Gamma\vdash w\,\lambda x\mapsto t : \qt qxS T}
\\
\Rule{\Gamma\vdash w\,f : \qt qxS T\qquad\Gamma\vdash \Q{q}w\;s : S}
  {\Gamma\vdash w\,f\,s : T[s/\V{x}]}
\end{array}\]

We will thus be able to type
\[\begin{array}{ll}
\F{vappend} : & \cat{\V{m}}{\D{Nat}}\cat{\V{n}}{\D{Nat}}\\
  & \pit{\V{xs}}{\D{Vec}\:\V{m}}\pit{\V{ys}}{\D{Vec}\:\V{n}}
   \D{Vec}\:(\V{m}\F{+}\V{n})
\end{array}\]

\newcommand{\word}{\overline}
Now, in the course of checking an
argument which itself is an application, the action of quantifiers on
worlds is \emph{iterated}. Accordingly, the worlds we can reach from a
given $w$ by iterated quantification or type formation form a
subcategory $\word{Q}\,w$ of $W$, `worlds \emph{local} to $w$', whose
objects are given by words in $\word{q}\in (Q\cup\{\Ty\})^\ast$, and
whose arrows $\word{Q}\,w(\word{q},\word{q'})$ are exactly the arrows
$\word{q}w\wto\word{q'}w$.

The key additional requirement which makes typing survive transmission
between worlds is that $\word{Q}$ is itself a functor from $W$ to
preorders. Whenever $v\wto w$, we need to map
$\word{Q}\,v(\word{q},\word{q'})$ to
$\word{Q}\,w(\word{q},\word{q'})$, so $\word{q}v\wto\word{q'}v$
implies $\word{q}w\wto\word{q'}w$. That is, embedding between worlds
preserves their \emph{local} preorder structure.


\section{Syntax and Computation}

Let us develop a more formal account of the situation. I propose to
adopt a bidirectional approach, explicitly separating those syntactic
forms from which types are synthesized and those for which types
proposed in advance are checked.
\newcommand{\neu}{\underline}
\newcommand{\Type}{\textsf{Type}}
\[\begin{array}{rrl}
s,t,R,S,T,U & ::= & \neu{e} \\
              &   |  & \Type \\
              &   |  & (q\,x\hb S)\to T  \\
              &   |  & \lambda x\mapsto t \medskip\\
e,f,E,F & ::= & (s\hb S) \\
              &   |  & x \\
              &   |  & f\:s
\end{array}\]
Observe that worlds in $W$ occur nowhere in the syntax of terms.
Quantifiers from $Q$, interpreted differently in different worlds, are
the things we see.

Substitution is defined as usual, but note that it makes sense only to
substitute $[e/x]$, not $[s/x]$. Moreover, $\lambda$ is
unencumbered by type annotation, but that any $\beta$-redex
necessarily requires a type annotation. Reduction is based on these
two rules:
\[
(\lambda x\mapsto t : (q\,x\hb S)\to T)\:s \leadsto t[(s\hb S)/x]
\qquad
\neu{(s\hb S)}\leadsto s
\]
Ultimately, we can eliminate type annotation from the system and insist
on computation in larger steps, by \emph{hereditary}
substitution. However, for now, let us work with a small-step system,
in order to develop the rest of the metatheoretical apparatus without
requiring the machinery of normalization. Indeed, to put my money
where my mouth is and also to reduce digression on the topic of
cumulative hierarchies (about which I have too much to say to say it
here), I shall work in the pleasingly na\"\i{}ve world of
$\Type$-in-$\Type$.

\newcommand{\cto}{\twoheadrightarrow}

Computation, $t \cto t'$, is the
reflexive-transitive contextual closure of $\leadsto$. It is
straightforward to show that $\cto$ is stable under substitution.
Moreover, as the contraction schemes contain no critical pairs, a
Takahashi-style `parallel reduction' argument shows that $\cto$ has
the `diamond property': if $R\cto S$ and $R\cto T$, then for some
$R'$, $S\cto R'$ and $T\cto R'$.


\section{Typing Rules}

Typing contexts look like this
\[
\Gamma,\Delta \;::=\; \cdot \;|\; \Gamma, w\,x\hb S
\]

Let us have judgment forms
\[
\Gamma \vdash \qquad
\Gamma \vdash w\; T \ni t \qquad
\Gamma \vdash w\; e \in S
\]

Context validity is as follows.
\[
\Axiom{\cdot\vdash}
\qquad
\Rule{\Gamma\vdash w\;\Type\ni S\quad w\in\Ty}
  {\Gamma,w\,x\hb S\vdash}
\]

I write $S\le T$ for the relation `an $S$ can be used wherever a $T$
is needed'. Locally, I define this by the rather symmetrical property
of computational joinability, $\exists U.\, S\cto U \wedge T\cto U$,
but we can imagine deploying standard methods to add $\eta$-rules. In
a cumulative setting, symmetry would evaporate, but a directed
inclusion is enough.

Type checking works like this:
\[\begin{array}{c}
\textbf{(typ)}\;\Rule{w\in\Ty\quad\Gamma\vdash}
  {\Gamma\vdash w\;\Type\ni\Type}
\qquad
\textbf{(fun)}\;\Rule{w\in\Ty\quad \Gamma,qw\,x\hb S \vdash w\;\Type\ni T}
  {\Gamma\vdash w\;\Type\ni(q\,x\hb S)\to T}
\\
\textbf{(syn)}\;\Rule{\Gamma \vdash w\; e\in S\quad S\le T}
   {\Gamma \vdash w\; T\ni\neu{e}}
\qquad
\textbf{(abs)}\;\Rule{\Gamma,qw\,x\hb S \vdash w\;T\ni t}
  {\Gamma\vdash w\;(q\,x\hb S)\to T\ni \lambda x\mapsto t}
\end{array}\]

Type synthesis works like this:
\[\begin{array}{c}
\textbf{(chk)}\;\Rule{\Gamma\vdash \Ty w\;\Type\ni S \quad
  S\cto T\quad \Gamma\vdash w\; T\ni s}
  {\Gamma \vdash w\; (s\hb S)\in S}
\qquad
\textbf{(var)}\;\Rule{\Gamma,u\,x\hb S,\Delta \vdash \qquad u\wto w}
  {\Gamma,u\,x\hb S,\Delta \vdash w\; x\in S}
\\
\textbf{(app)}\;\Rule{\Gamma\vdash w\; f\in R\quad R \cto (q\,x\hb S)\to T \quad
         \Gamma\vdash qw\; S\ni s}
  {\Gamma\vdash w\; f\:s \in T[(s\hb S)/x]}
\end{array}\]

Context validity amounts to this:
\[
\textbf{(emp)}\;\Axiom{\quad\vdash\quad}
\qquad
\textbf{(ext)}\;\Rule{\Gamma\vdash \Ty u\;\Type\ni S}
  {\Gamma,u\,x\hb S\vdash}
\]

It is my habit to let the metavariable $J$ range over anything which
may extend a context to a judgment and to write $\Gamma\vdash J$ for
that judgment. E.g., if $J = \Delta\vdash$, $\Gamma\vdash J$ is the
judgment $\Gamma,\Delta\vdash$.

The following sanity property is admissible, by straightforward
induction on derivations.
\[
\Rule{\Gamma\vdash J}
  {\Gamma \vdash}
\]

With the usual remarks about variable freshness, we also have
thinning admissible,
\[
\Rule
 {\Gamma \vdash w\;\Type\ni S\quad w\in\Ty\quad\Gamma\vdash J}
 {\Gamma,u\,x\hb S\vdash J}
\]
and thence a chance to establish stability under substitution by the
usual method of glorified substitution
\[
\Rule
 {\Gamma,u\,x\hb S\vdash J \quad S\cto T\quad \Gamma\vdash u\; T\ni s}
 {\Gamma\vdash J[(s\hb S)/x]}
\]
simply because wherever in a derivation the variable rule is used, we
shall be able to replace it by a suitably thinned synthesis for $(s\hb
S)\in S$. The catch, however, is that the variable can be transported
up the world preorder. We must show that its replacement can be
correspondingly transported.

It might be nice to show the following admissible (and indeed they are)
\[
\Rule{\Gamma \vdash v\;T\ni t\quad v\wto w}
  {\Gamma \vdash w\;T\ni t}
\qquad
\Rule{\Gamma \vdash v\;e \in S \quad v\wto w}
  {\Gamma \vdash w\;e\in S}
\]
but the obvious strategy of induction on derivations will fail as soon
as we try to go under a binder, putting $v$s and $w$s into the
context.

We need a suitable generalization to make the induction go
through. The key is to consider judgments-with-world-holes, $J(\cdot)$
where some quantifier-word-applied $\word{q}\cdot$ can stand as a
world: in particular, let us insist that the target world of a typing
judgment is of that form. We can form a proper judgment $J(w)$ by
substituting $w$ for the $\cdot$ and computing the iterated quantifier
action.

\textbf{Lemma (world subsumption).~} The following is admissible.
\[
\Rule{\Gamma\vdash J(v)\quad v\wto w}
     {\Gamma\vdash J(w)}
\]
\textbf{Proof.~} Induction on derivations. Let us proceed rule by
rule.
\begin{description}
\item[emp] $J(\cdot)$ is $\vdash$, so the assumption and conclusion coincide
\item[ext] there is nothing to prove unless
  $J(\cdot)=\Delta(\cdot),\word{q}\cdot\,x\hb S\vdash$; if so, we must have
  had the premise $\Gamma,\Delta(v)\vdash \Ty\word{q}v\;\Type\ni S$, so the inductive
  hypothesis tells us that $\Gamma,\Delta(w)\vdash \Ty\word{q}w\;\Type\ni
  S$, and we may deduce that $\Gamma,\Delta(w)\vdash
  \Ty\word{q}w\;\Type\ni S$
\item[typ] we have
  $\Gamma,\Delta(v)\vdash\word{q}v\;Type\ni\Type$, so we must
  have had $\word{q}v\in\Ty$ and $\Gamma,\Delta(v)\vdash$;
  $\word{q}\cdot$ is monotonic and $\Ty$
  is an upset, so $\word{q}w\in\Ty$; by inductive hypothesis,
  $\Gamma,\Delta(w)\vdash$; we thus deduce
  $\Gamma,\Delta(w)\vdash\word{q}w\;\Type\ni\Type$
\item[fun] we have
  $\Gamma,\Delta(v)\vdash\word{q}v\;\Type\ni(q\,x\hb S)\to T$, so we
  must have had $\Gamma,\Delta(v),q\word{q}v\,x\hb S\vdash\;\Type\ni
  T$, with $\word{q}v\in\Ty$; as before, $\word{q}w\in\Ty$ and
  by induction $\Gamma,\Delta(w),q\word{q}w\,x\hb S\vdash\;\Type\ni
  T$; hence $\Gamma,\Delta(w)\vdash\word{q}w\;\Type\ni(q\,x\hb
  S)\to T$
\item[syn] induction
\item[abs] induction---as with \textbf{fun}, generalizing to
  $J(\cdot)$ lets us bind a variable in world $q\word{q}\cdot$
\item[chk] induction
\item[var] there are two cases
  \begin{enumerate}
  \item $u\,x\hb S$ in $\Gamma,\Delta(\cdot)$ with no $\cdot$ in $u$;
    $\Gamma,\Delta(v)\vdash\word{q}v\;x\in S$\\
    we must have $\Gamma,\Delta(v)\vdash$ and $u\wto
    \word{q}v$; by induction, $\Gamma,\Delta(w)\vdash$;
    monotonicity tells us that $\word{q}v\wto\word{q}w$,
    so by transitivity, $u\wto\word{q}w$; we may conclude
    $\Gamma,\Delta(w)\vdash\word{q}w\;x\in S$
  \item $\word{q'}\cdot\,x\hb S$ in $\Gamma,\Delta(\cdot)$;
    $\Gamma,\Delta(v)\vdash\word{q}v\;x\in S$\\
    we must have $\Gamma,\Delta(v)\vdash$ and $\word{q'}v\wto
    \word{q}v$; by induction, $\Gamma,\Delta(w)\vdash$;
    functoriality of $\word{Q}$ tells us that
    $\word{q'}w\wto \word{q}w$; we may conclude
    $\Gamma,\Delta(w)\vdash\word{q}w\;x\in S$
  \end{enumerate}
\item[app] induction \hfill $\square$
\end{description}

We now obtain stability under substitution, as stated above, by a
straightforward induction on derivations. The crux comes when we are
given $\Gamma,u\,x\hb S,\Delta\vdash w\;x\in S$ with $u\wto v$, $S\cto T$ and
$\Gamma\vdash u\;T\ni s$ and asked to deduce
$\Gamma,\Delta[s:S/x]\vdash w\;s:S\in S$. We have just what we need to
establish $\Gamma\vdash u\;s:S\in S$, then world subsumption and
thinning finish the job.

\section{Detritus}

I'll write this when I know what I'm talking about. I shall probably
cite~\cite{brady:indices}.



\bibliographystyle{plainnat} % basic style, author-year citations
\bibliography{Ornament} % name your BibTeX data base


\end{document}